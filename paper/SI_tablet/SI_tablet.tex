% Template for Cogsci submission with R Markdown

% Stuff changed from original Markdown PLOS Template
\documentclass[10pt, letterpaper]{article}

\usepackage{cogsci}
\usepackage{pslatex}
\usepackage{float}
\usepackage{caption}

% amsmath package, useful for mathematical formulas
\usepackage{amsmath}

% amssymb package, useful for mathematical symbols
\usepackage{amssymb}

% hyperref package, useful for hyperlinks
\usepackage{hyperref}

% graphicx package, useful for including eps and pdf graphics
% include graphics with the command \includegraphics
\usepackage{graphicx}

% Sweave(-like)
\usepackage{fancyvrb}
\DefineVerbatimEnvironment{Sinput}{Verbatim}{fontshape=sl}
\DefineVerbatimEnvironment{Soutput}{Verbatim}{}
\DefineVerbatimEnvironment{Scode}{Verbatim}{fontshape=sl}
\newenvironment{Schunk}{}{}
\DefineVerbatimEnvironment{Code}{Verbatim}{}
\DefineVerbatimEnvironment{CodeInput}{Verbatim}{fontshape=sl}
\DefineVerbatimEnvironment{CodeOutput}{Verbatim}{}
\newenvironment{CodeChunk}{}{}

% cite package, to clean up citations in the main text. Do not remove.
\usepackage{cite}

\usepackage{color}

% Use doublespacing - comment out for single spacing
%\usepackage{setspace}
%\doublespacing


% % Text layout
% \topmargin 0.0cm
% \oddsidemargin 0.5cm
% \evensidemargin 0.5cm
% \textwidth 16cm
% \textheight 21cm

\title{A speed-accuracy tradeoff in children's processing of scalar
implicatures}


\author{{\large \bf Rose M. Schneider} \\ \texttt{rschneid@stanford.edu} \\ Department of Psychology \\ Stanford University \And {\large \bf Michael C. Frank} \\ \texttt{mcfrank@stanford.edu} \\ Department of Psychology \\ Stanford University}

\begin{document}

\maketitle

\begin{abstract}
Scalar implicatures---inferences from a weak description (``I ate some
of the cookies'') that a stronger alternative is not true (``I didn't
eat all'')---are a paradigm case of pragmatic inference. Children's
trouble with scalar implicatures is thus an important puzzle for
theories of the development of pragmatics, given their communicative
competence in other domains. Here, we explore children's reaction times
in a new paradigm for measuring scalar implicature processing. Alongside
falures on scalar implicature with ``some,'' we replicate previous
reports of failures on the quantifier ``none'' and find evidence of a
speed-accuracy tradeoff for both ``some'' and ``none.'' Motivated by
these findings, we use a Drift Diffusion Model to explore the
relationship between accuracy and reaction time in our task and find
evidence consistent with the hypothesis that preschoolers lack access to
the relevant alternatives for the scalar implicature computation. The
set of relevant alternatives may be broader than has been previously
assumed, however.

\textbf{Keywords:}
Pragmatics; development; scalar implicature; diffusion models.
\end{abstract}

\section{Introduction}\label{introduction}

Language comprehension in context is an inferetial process. Listeners
are not limited to interpreting the literal meaning of speakers'
utterances; they can also reason about what the speaker intended, based
on other utterances the speaker could have said. In the case of
\emph{pragmatic implicatures} (Grice, 1975), a speaker employs a weaker
literal description to imply that a stronger alternative is true. Adult
listeners tend to infer from the statement ``I enjoyed \emph{some} of my
winter break'' that some, but not all, of the break was pleasant. This
\emph{scalar implicature} (SI) relies heavily on a knowledge of the
relevant lexical alternatives in the quantifier scale \(<\)\emph{some},
\emph{all}\(>\). On standard theories, a listener must be able to
contrast ``some'' with the stronger descriptor ``all'' to compute the
implicature (Grice, 1975,Levinson (2000)).

SIs are challeinging for children until surprisingly late in development
(Noveck, 2001). For example, when judging a scene in which three of
three horses have jumped over a fence, five-year-olds are likely to
endorse the statement ``some of the horses jumped over the fence'' as
felicitous, despite the presence of a more informative alternative
(``all''; Papafragou \& Musolino, 2003). Children do seem to have some
knowledge of these scalar terms, however; for example, they reward
speakers based on the informativeness of their scalar descriptions
(Katsos \& Bishop, 2011). Given this early sensitivity, why do children
still struggle to compute scalar implicatures until late in development?

One possibile cause of children's failures is that they might not have
access to the relevant lexical alternatives (D. Barner \& Bachrach,
2010). This idea, which we will refer to as the \emph{Alternatives
Hypothesis}, predicts that if children cannot quickly and reliabily
bring to mind the relevant alternative quantifiers (e.g., ``all'' in a
situation where they hear ``some'') they will be unable to make the
implicature computation. The alternatives hypothesis makes a number of
predictions about children's abilities in reasoning about quantifiers,
some of which have been confirmed empirically. For example, consistent
with the idea of inaccessible alternatives, David Barner, Brooks, \&
Bale (2011) showed that four-year-olds could not even resolve the
quantifier expression ``only some'' (which should force alternatives to
be negated semantically, rather than pragmatically). But what are the
proper alternatives for SIs?

With respect to the proper set of alternatives for SI, the empirical
evidence has been changing rapidly. Although the conventional view on SI
is that the primary inferential alternative is ``all,'' a new body of
evidence suggests that more alternatives may be necessary---in
particular, ``none.'' For example, Degen \& Tanenhaus (2015) found that
set size changes the felicity of quantifier SIs for adults: ``some'' is
more felicitous when you couldn't say ``one'' or ``two.'' In a
computational reanalysis of these and other data, Franke (2014) showed
that a high weight on the alternative ``none'' was critical for fitting
these data. And in a recent study with children, Skordos \& Papafragou
(in press) found that exposing children to either ``all'' \emph{or}
``none'' facilitated their computation of subsequent SIs.

This relationship to ``none'' is unexpected on classic Gricean theories
(Grice, 1975, Horn (1972)), where the only alternatives should be those
logically entailed by the original message (i.e. ``all''). But it
\emph{is} in fact predicted by recent probabilistic models of
implicature. Under these models, all the relevant alternatives compete
with one another (Goodman \& Stuhlm{ü}ller, 2013, Franke (2014)). On the
other hand, all of the evidence cited above for the claim of ``none'' as
an alternative is relatively indirect, and such a substantial revision
to standard theory requires further evidence.

\begin{CodeChunk}
\begin{figure}[b]

{\centering \includegraphics{figs/image-1} 

}

\caption[Example trial stimuli used in Horowitz and Frank (2015)]{Example trial stimuli used in Horowitz and Frank (2015).}\label{fig:image}
\end{figure}
\end{CodeChunk}

One other recent developmental study further supports the importance of
``none'' in SIs and provides the starting point for our current
experiment. Horowitz \& Frank (2015) designed a referent selection
paradigm that could be used across a broad age range (3--5 years) to
explore both scalar and ad-hoc (context dependent) implicatures. In this
task, children saw three book covers, each featuring four familiar
objects (Figure \ref{fig:image}). On target trials, the experimenter
described a book using a semantically ambiguous description (e.g., ``On
the cover of my book, some of the pictures are cats'' {[}scalar{]} or
``On the cover of my book are cats'' {[}ad hoc{]}). Children succeded on
ad-hoc trials but largely failed to make SIs, suggesting they had the
pragmatic competence necessary to compute the implicature.

Interestingly, in Horowitz \& Frank (2015), the same children who failed
on SI also failed on unambiguous ``none'' control trials---and in
several samples, performance was highly correlated between ``none'' and
``some'' trials. This result would be predicted if ``none'' was in fact
an inferential alternative. If some children were not computing its
semantics appropriately in an online fashion, they would be the children
to fail in the SI computation as well.

One further prediction of the alternatives hypothesis relates to
processing time. Perhaps children who have a fully-established
quantifier scale---and hence can make correct SIs---take additional time
in using this information, due to competition between alternatives.
Congruent with this prediction, our intuition in the Horowitz \& Frank
(2015) study described above and in pilot studies using this same
paradigm was that when children made correct SIs they appeared to be
taking longer than when they failed. But although reaction time measures
have been commonplace in studies of adults' SI processing, they have
been almost entirely absent in the developmental literature (with the
exception of Huang \& Snedeker, 2009, whose data showed little evidence
of SI computation).

Thus, in our current study, we explore children's behavioral response
latencies in an iPad adaptation of the Horowitz \& Frank (2015) scalar
implicature task. In our analyses, we explore overall accuracy and
patterns of performance, as in (Horowitz \& Frank, 2015), and find that
children not only struggle in making a scalar implicature, but replicate
the finding that they also grapple with ``none'' until fairly late in
development. Congruent with our predictions, in examining reaction time
patterns across all quantifier types, we find evidence of a
speed-accuracy tradeoff for both quantifiers. Finally, we use a Drift
Diffusion Model to explore the source of this increased reaction time.
Overall, our findings are consistent with a version of the Alternatives
Hypothesis under which ``none'' is an important inferential alternative
in SI and its availability causes slower processing times but correct
SIs. We consider this and other alternative explanations in the
Discussion.

\section{Method}\label{method}

In this study, we adapted the scalar implicature paradigm developed by
Horowitz \& Frank (2015) for the iPad. In addition to capturing detailed
reaction time data, this version included more trials, and standardized
prosody across all trials, as well as a completely randomized
design.\footnote{The full experiment can be viewed online at \texttt{https://rosemschneider.github.io/tablet\_exp/si\_tablet.html} and all of our data, processing, experimental stimuli, and analysis code can be viewed in the version control repository for this paper at: \texttt{https://github.com/rosemschneider/SI\_tablet}.}

\subsection{Participants}\label{participants}

\begin{table}[t]
\centering
\begin{tabular}{c c c c c } 
 \hline
 Age group & N & Mean & Median & SD \\
 \hline
 3--3.5 years & 24 & 3.27 & 3.27 & 0.14\\
 3.5--4 years & 35 & 3.78 & 3.73 & 0.15 \\ 
 4--4.5 years & 25 & 4.28 & 4.28 & 0.15\\
 4.5--5 years & 30 & 4.76 & 4.76 & 0.15 \\
 5--6.5 years & 24 & 5.55 & 5.56 & 0.36 \\
 \hline
\end{tabular}
\caption{Age information for all participants.}
\label{tab:age}
\end{table}

Table \ref{tab:age} shows the breakdown of age information for all
participants. Included in analyses are 138 children out of a planned
sample of 120 participants, recruited from both a local daycare and a
local children's museum. We ran 20 additional children, who were
excluded from analysis based on planned exclusion criteria of low
English language exposure (\(\leq 75\%\)) or \(<50\%\) of trials
completed. Included in our sample were 79 females and 59
males.\footnote{Based on Horowitz \& Frank (2015), we initially planned
  to collect data from children 3--5 years. After collecting data from
  57 participants, however, we observed significantly lower performance
  on implicature trials across all age groups, indicating that the iPad
  adaptation of the scalar implicature task was slightly more
  challenging for all children, and included an older age group of 24
  5--6.5-year-olds.}

\subsection{Stimuli and design}\label{stimuli-and-design}

The general format of the task was identical to Horowitz \& Frank
(2015), with the exception of added items for additional trials. The
study was programmed in HTML, CSS, and JavaScript, and displayed to
children on a full-sized iPad. Each trial displayed three book covers,
each containing a set of four familiar objects (Figure \ref{fig:image}).
Each session involved 30 trials, with 10 trials per quantifier-type
(``all'', ``some'', and ``none''). Each audio clip used the same three
initial sentence frames (e.g., ``On the cover of my book, \emph{some} of
the pictures\ldots{}'') so that prodosdy was emphasized equally across
all trials. The average length of each audio clip (including target item
phrase, e.g., ``\ldots{}are cats'') was approximately 6s. In our
randomization, quantifier triad order, items (within category), target
item, and quantifier were randomized for all participants. In all, there
were 270 different target items and audio clips.

\subsection{Procedure}\label{procedure}

Sessions took place individually in a small testing room away from the
museum floor or the classrooom of the daycare. To familiarize children
with the iPad, each session began with a ``dot game,'' which required
them to press dots on the screen as fast as possible. After the dot
game, the experimenter introduced them to ``Hannah,'' a cartoon
character who wanted to play a guessing game with her books. The
experimenter explained that Hannah would show the child three books, and
would give one hint about which book she had in mind, so they had to
listen carefully. Children then saw a practice trial with an unambiguous
noun referent.

Each trial allowed 2.5s for children to visually inspect the three book
covers, before the experiment played the trial prompt (e.g., ``On the
cover of my book, \emph{none} of the pictures are cats.''). Reaction
times were measured from the onset of the target word. Children could
only make one selection. If a child was not paying attention, or if she
did not hear Hannah's prompt, the experimenter repeated it, matching the
original prosody. Once children correctly made their selection, a green
box appeared around the chosen book. The experiment was self-paced, and
children initiated each trial by pressing a button that appeared after
they had made their selection in the previous trial.

\begin{CodeChunk}
\begin{figure}[t]
\includegraphics{figs/overall_acc-1} \caption[Children's overall accuracy for each quantifier type]{Children's overall accuracy for each quantifier type. Bars show mean performance for each age group. Error bars are 95 percent confidence intervals computed by non-parametric bootstrap.}\label{fig:overall_acc}
\end{figure}
\end{CodeChunk}

\section{Results}\label{results}

To exclude trials where the child had missed the prompt or was not
paying attention, we excluded reaction times (RTs) longer than 15s.
After this initial cut, we excluded RTs outside three standard
deviations of the log of mean reaction time. This cleaning process
resulted in a data loss of 85 trials (2.14\%).

\subsection{Accuracy}\label{accuracy}

Figure \ref{fig:overall_acc} shows children's for each trial type, split
by age group. For each age group, we saw significantly lower accuracy
for the quantifiers ``some'' and ``none'' in comparison to ``all'' (all
\(p\)s \(< .01\) in two-sample t-tests for each age group). These
results generally replicate our previous findings using this paradigm
(Horowitz \& Frank, 2015), but one difference from the previous results
was in implicature trials. Children aged 3--5 years performed
significantly lower on ``some'' (implicature) trials in this task in
comparison with data from Horowitz, Schneider, \& Frank (in prep.)
(\(p < .01\) for all tests). Thus, while the iPad adaptation was
generally successful, implicatures were more difficult, perhaps because
of the non-social nature of the iPad interaction or the recorded audio
stimuli.

\begin{CodeChunk}
\begin{figure}[t]
\includegraphics{figs/diptest-1} \caption[Frequency histogram of correct responses for each trial type, across all participants]{Frequency histogram of correct responses for each trial type, across all participants.}\label{fig:diptest}
\end{figure}
\end{CodeChunk}

\begin{CodeChunk}
\begin{figure*}[t]

{\centering \includegraphics{figs/dense-1} 

}

\caption[Density plots of reaction times for correct and incorrect responses on each trial type, split by age]{Density plots of reaction times for correct and incorrect responses on each trial type, split by age.}\label{fig:dense}
\end{figure*}
\end{CodeChunk}

We next fit a logistic mixed effects model predicting correct response
as an interaction of age and trial type, with random effects of trial
type and
participant.\footnote{All mixed effects models were fit in \texttt{R} using the \texttt{lme4} package. The model specification was: \texttt{correct ~ age * trial type + (trial type | subject id)}.}
Performance was significantly lower on ``some'' (\(\beta = -6.98\),
\(p < .0001\)) and ``none'' trials (\(\beta = -9.55\), \(p < .0001\)).
There was also a signficant interaction between age and trial type on
``none'' trials (\(\beta\) = 1.53, \(p\) \textless{} .0001), indicating
that children's performance with this difficult quantifier increased
with age.

Figure \ref{fig:diptest} shows distributions of correct responses for
all trial types. Performance on ``some'' and ``none'' trials was bimodal
(Hartigan's \(D\) = 0.08, \(p\) \textless{} .0001) and ``none'' trials
(\(D\) = 0.11, \(p\) \textless{} .0001). While children's average
accuracy was low for these quantifiers, there were some children who
were correct on the majority of these trials (``Some'': N = 28;
``None'': N = 36) and the others were typically incorrect on the
majority of trials. Children did not appear to be responding randomly.
As in previous work, we found a strong correlation between children's
accuracy on ``some'' and ``none'' trials (\(r\) = 0.49, \(p\)
\textless{} .0001).

\subsection{Reaction time}\label{reaction-time}

We fit a linear mixed effects model predicting log RT on correct trials
as a function of log trial number, the interaction of age and trial
type, and random effects of trial type by subject.\footnote{Model
  specification:
  \texttt{log(reaction time) ~ log(trial number) + age * trial type + (trial type | subject id)}.
  Age was centered for ease of interpretation of coefficients, and we
  calculated \emph{p} values via the \(t=z\) approximation.} Reaction
times were longer on ``none'' (\(\beta = 0.22\), \(p < .0001\)) and
``some'' trials (\(\beta = 0.1\), \(p < .0001\)), and reaction times
decreased with age (\(\beta = -0.1\), \(p < .0001\)). There were no
significant interactions between age and trial type. The model also
showed a main effect of trial number, with reaction times decreasing
over the course of the study (\(\beta\) = -0.27, \(p\) \textless{}
.00001).

Examination of the pattern in Figure \ref{fig:dense} suggests that
accuracy and reaction time may be interacting, however. In particular,
while correct responses on ``all'' trials appear to be faster than the
(few) incorrect responses, the opposite is true for ``none'' and
``some'' trials: Errors have faster RTs, potentially indicating a
speed-accuracy tradeoff. To test for this effect, we fit another mixed
effects model, this time including accuracy and its interactions with
age and trial type as predictors. This model revealed that correct
trials overall had faster RTs (\(\beta = -0.16\), \(p = .0002\)), but
that this accuracy term interacted negatively with trial type such that
both ``none'' and ``some'' trials had slower RTs for correct trials
(\(\beta = 0.34\), \(p < .0001\); \(\beta = 0.27\), \(p < .0001\)).
There were no three-way interactions of trial-type and age. This model
thus provides evidence of a speed-accuracy tradeoff for ``some'' and
``none'' trials.

We fit a linear mixed effects model predicting log RT on correct trials
as a function of log trial number, the interaction of age and trial
type, and random effects of trial type by subject.\footnote{Model
  specification:
  \texttt{log(reaction time) ~ log(trial number) + age * trial type + (trial type | subject id)}.
  Age was centered for ease of interpretation of coefficients, and we
  calculated \emph{p} values via the \(t=z\) approximation.} Reaction
times were longer on ``none'' (\(\beta = 0.38\), \(p < .0001\)) and
``some'' trials (\(\beta = 0.22\), \(p < .0001\)), and reaction times
decreased with age (\(\beta = -0.29\), \(p < .0001\)). There were no
significant interactions between age and trial type. The model also
showed a main effect of trial number, with reaction times decreasing
over the course of the study (\(\beta\) = -0.1, \(p\) \textless{}
.00001).

Examination of the pattern in Figure \ref{fig:dense} suggests that
accuracy and reaction time may be interacting, however. In particular,
while correct responses on ``all'' trials appear to be faster than the
(few) incorrect responses, the opposite is true for ``none'' and
``some'' trials: Errors have faster RTs, potentially indicating a
speed-accuracy tradeoff. To test for this effect, we fit another mixed
effects model, this time including accuracy and its interactions with
age and trial type as predictors. This model revealed that correcvt
trials overall had faster RTs (\(\beta = -0.16\), \(p = .0002\)), but
that this accuracy term interacted negatively with trial type such that
both ``none'' and ``some'' trials had slower RTs for correct trials
(\(\beta = 0.34\), \(p < .0001\); \(\beta = 0.27\), \(p < .0001\)).
There were no three-way interactions of trial-type and age. This model
thus provides evidence of a speed-accuracy tradeoff for ``some'' and
``none'' trials.

\subsection{Drift diffusion models}\label{drift-diffusion-models}

\begin{CodeChunk}
\begin{figure*}[t]

{\centering \includegraphics{figs/devo_param_plot-1} 

}

\caption[Parameter estimates for drift diffusion model, split by age and trial type]{Parameter estimates for drift diffusion model, split by age and trial type. Error bars are 95 percent confidence intervals computed by nonparametric bootstrap.}\label{fig:devo_param_plot}
\end{figure*}
\end{CodeChunk}

Motivated by the evidence of a speed-accuracy tradeoff we observed, we
further explored the interaction between reaction time and accuracy in
more depth using drift diffusion modeling. DDM can be used in behavioral
tasks to provide a more detailed view of the relationship between
accuracy and reaction time (Ratcliff \& Rouder, 1998). In DDM, a
behavioral response (a correct or incorrect choice) is the result of
noisy data accumulation through a diffusion process. Responses have
\emph{separation boundaries} that are dependent on the amount of
information needed to initate a response, and \emph{drift rate}
formalizes the rate of data accumulation. \emph{Nondecision} is the
amount of time between stimuli offset, and initiating the diffusion
process. Finally, different responses may have a \emph{bias}, or
different starting point in the diffusion process, dependent on the
stimuli.

\subsubsection{Developmental analyses}\label{developmental-analyses}

Although DDMs are traditionally fit to data from two-alternative
forced-choice tasks, here we estimate the drift process between a
correct and incorrect choice, with two options in each trial being
``incorrect,'' and only one being consistent with the target noun and
quantifier. We estimated parameters for each subject for each trial type
using the \texttt{RWiener} package. We then aggregated across subjects
to obtain means and confidence intervals for each age group. Figure
\ref{fig:devo_param_plot} shows the parameter estimates for each age
group, split by trial type.

For each estimate, we ran a mixed effects model, predicting parameter
value as an interaction of age and trial
type.\footnote{The specifications for all parameter models are as follows: \texttt{Parameter Value ~ age * trial type + (1 | subject ID)}}
For boundary separation, there was no significant effect of trial type,
indicating that roughly the same amount of information needs to be
accumulated to make a decision in each trial type. For nondecision time,
we found a significant main effect of age (\(\beta\) = -0.28, \(p\)
\textless{} .00001), as well as a interaction between age and ``none''
trials (\(\beta\) = 0.24, \(p\) = .01). As expected in drift rate, there
was a negative main effect of trial type (``None'': \(\beta\) = -1.3,
\(p\) = .0185; ``Some'': \(\beta\) = -1.18, \(p\) = .03). Interestingly,
for bias there was a significant negative effect of ``none'' trials
(\(\beta\) = -0.5, \(p\) = .0005), and ``some'' trials showed a trend
towards significance (\(\beta\) = -0.28, \(p\) = .0503), as well as a
significant interaction between age and ``none'' trials (\(\beta\) =
0.07, \(p\) = .023). The parameter estimates from our DDM align with the
analyses presented above: Older children, who are more familiar with the
quantifier scale, are more likely to respond correctly in our scalar
implicature task, while younger children's failures appear to be due to
a low rate of data accumulation and a high separation boundary.

\subsubsection{Exploratory analyses}\label{exploratory-analyses}

In addition to examining effects of age on the dyiffusion process, we
additionally conducted an exploratory analysis, examining differences in
the decision-making process for children who consistently made SIs
compared with those who did not. We split children by accuracy on scalar
implicature trials, and then estimated parameters by accuracy group.
High accuracy was defined as an average of 75\% or higher performance on
scalar implicature trials. Figure \ref{fig:param_plot} shows parameter
estimates for each accuracy group, split by trial type.

\begin{CodeChunk}
\begin{figure*}[t]

{\centering \includegraphics{figs/param_plot-1} 

}

\caption[Parameter estimates for drift diffusion model, split by accuracy and trial type]{Parameter estimates for drift diffusion model, split by accuracy and trial type. Error bars are 95 percent confidence intervals computed by nonparametric bootstrap.}\label{fig:param_plot}
\end{figure*}
\end{CodeChunk}

We again used mixed-effects models to predict DDM coefficients across
participants. As in the developmental DDM analysis, there were no
significant effects of separation or nondecision. And while drift rates
showed a significant effect of accuracy, because we estimated parameters
for high- and low-accuracy children separately, these differences are
expected. In our bias estimates, however, we found a significant
interaction between accuracy group and trial type on ``some'' trials
(\(\beta\) = -0.18, \(p = .0013\)). This interaction suggests that bias
(the starting point in the diffusion process) might be an important
factor in successfully making a scalar implicature: More successful
children were less biased towards incorrect response alternatives.

\section{General Discussion}\label{general-discussion}

What makes scalar implicatures using quantifiers so hard for children?
The best current hypothesis posits that children do not have access to
the appropriate inferential alternatives and hence fail to consider them
in their pragmatic computation (D. Barner \& Bachrach, 2010; David
Barner et al., 2011). But what are those alternatives? A variety of
recent work has suggested that the negative alternative ``none'' may
compete with ``some'' and ``all.'' Our findings here are consistent with
this account and provide some additional support. We replicated the
pattern found in previous studies that those children who succeed in
comprehending the quantifier ``none'' also make SIs (Horowitz \& Frank,
2015; Horowitz et al., in prep.). In addition, our data revealed a
speed-accuracy tradeoff, such that reaction times in those trials in
which children succeeded in making SIs were slower overall.

One interpretation of this speed-accuracy tradeoff is that children who
have more inferential alternatives accessible to them (e.g.~are
considering ``none,'' ``some,'' and ``all'' together) are both better at
making SIs and slower to make them due to the processing cost of making
the inference. Our data are consistent with this account, and we also
found some evidence in favor of it from an exploratory drift diffusion
model analysis. We fit a DDM to our data for children who succeeded in
making scalar implicatures versus children who fail. The model suggested
that bias in ``some'' and ``none'' trials might be a key factor related
to success---that is, children who were considering ``some'' and ``all''
responses equally in their decision were more likely to make the SI.
This finding again is consistent with the idea that weighing
alternatives appropriately in the SI computation is critical to success.

The speed-accuracy patterns we report are correlational, however, and
other accounts are consistent with them as well. For example, some third
factor (say inhibitory control) could underly the ability to succeed in
``some'' and ``none'' trials and also explain why some children are able
to inhibit their response long enough to complete the SI computation.
Horowitz et al. (in prep.) did not find evidence of correlations between
individuals' SI abilities and their executive function using one popular
measure (the dimensional change card sort). Other versions of this
account (or other accounts entirely) are still possible, however.
Nevertheless, our work here suggests that there is a meaningful
relationship between children's accuracy and processing times in making
scalar implicatures.

\section{Acknowledgements}\label{acknowledgements}

Thanks to Bing Nursery School and the San Jose Children's Discovery
Museum. Thanks also to Veronica Cristiano, Rachel Walker, and Tamara
Mekler for their help with data collection, and to Kara Weisman and Ann
Nordmeyer for their assistance creating stimuli.

\section{References}\label{references}

\setlength{\parindent}{-0.1in} \setlength{\leftskip}{0.125in} \noindent

Barner, D., \& Bachrach, A. (2010). Inference and exact numerical
representation in early language development. \emph{Cognitive
Psychology}, \emph{60}(1), 40--62.

Barner, D., Brooks, N., \& Bale, A. (2011). Accessing the unsaid: The
role of scalar alternatives in children’s pragmatic inference.
\emph{Cognition}, \emph{118}(1), 84--93.

Degen, J., \& Tanenhaus, M. K. (2015). Processing scalar implicature: A
constraint-based approach. \emph{Cognitive Science}, \emph{39}(4),
667--710.

Franke, M. (2014). Typical use of quantifiers: A probabilistic speaker
model. In \emph{Proceedings of the 36th annual conference of the
cognitive science society} (pp. 487--492).

Goodman, N. D., \& Stuhlm{ü}ller, A. (2013). Knowledge and implicature:
Modeling language understanding as social cognition. \emph{Topics in
Cognitive Science}, \emph{5}(1), 173--184.

Grice, H. P. (1975). Logic and conversation. In P. Cole \& J. Morgan
(Eds.), \emph{Syntax and semantics} (Vol. 3). New York: Academic Press.

Horn, L. R. (1972). \emph{On the semantic properties of logical
operators.} (PhD thesis). University of California, Los Angeles.

Horowitz, A., \& Frank, M. C. (2015). Sources of developmental change in
pragmatic inferences about scalar terms. In \emph{Proceedings of the
37th annual conference of the cognitive science society.}

Horowitz, A., Schneider, R. M., \& Frank, M. C. (in prep.). The trouble
with quantifiers: Children's difficulties with ``some'' and ``none''.

Huang, Y. T., \& Snedeker, J. (2009). Online interpretation of scalar
quantifiers: Insight into the semantics-pragmatics interface.
\emph{Cognitive Psychology}, \emph{58}(3), 376--415.

Katsos, N., \& Bishop, D. (2011). Pragmatic tolerance: Implications for
the acquisition of informativeness and implicature. \emph{Cognition},
\emph{120}(1), 67--81.

Levinson, S. C. (2000). \emph{Presumptive meanings: The theory of
generalized conversational implicature}. MIT Press.

Noveck, I. (2001). When children are more logical than adults:
Experimental investigations of scalar implicature. \emph{Cognition},
\emph{78}(2), 165--188.

Papafragou, A., \& Musolino, J. (2003). Scalar implicatures: Experiments
at the semantics-pragmatics interface. \emph{Cognition}, \emph{86}(3),
253--282.

Ratcliff, R., \& Rouder, J. (1998). Modeling response times for
two-choice decisions. \emph{Psychologial Science}, \emph{9}(5),
347--356.

Skordos, D., \& Papafragou, A. (in press). Children's derivation of
scalar implicatures: Alternatives and relevance. \emph{Cognition}.

\end{document}
