% Template for Cogsci submission with R Markdown

% Stuff changed from original Markdown PLOS Template
\documentclass[10pt, letterpaper]{article}

\usepackage{cogsci}
\usepackage{pslatex}
\usepackage{float}

% amsmath package, useful for mathematical formulas
\usepackage{amsmath}

% amssymb package, useful for mathematical symbols
\usepackage{amssymb}

% hyperref package, useful for hyperlinks
\usepackage{hyperref}

% graphicx package, useful for including eps and pdf graphics
% include graphics with the command \includegraphics
\usepackage{graphicx}

% Sweave(-like)
\usepackage{fancyvrb}
\DefineVerbatimEnvironment{Sinput}{Verbatim}{fontshape=sl}
\DefineVerbatimEnvironment{Soutput}{Verbatim}{}
\DefineVerbatimEnvironment{Scode}{Verbatim}{fontshape=sl}
\newenvironment{Schunk}{}{}
\DefineVerbatimEnvironment{Code}{Verbatim}{}
\DefineVerbatimEnvironment{CodeInput}{Verbatim}{fontshape=sl}
\DefineVerbatimEnvironment{CodeOutput}{Verbatim}{}
\newenvironment{CodeChunk}{}{}

% cite package, to clean up citations in the main text. Do not remove.
\usepackage{cite}

\usepackage{color}

% Use doublespacing - comment out for single spacing
%\usepackage{setspace}
%\doublespacing


% % Text layout
% \topmargin 0.0cm
% \oddsidemargin 0.5cm
% \evensidemargin 0.5cm
% \textwidth 16cm
% \textheight 21cm

\title{A speed-accuracy tradeoff in children's processing of scalar
implicatures}


\author{{\large \bf Rose M. Schneider} \\ \texttt{rschneid@stanford.edu} \\ Department of Psychology \\ Stanford University \And {\large \bf Michael C. Frank} \\ \texttt{mcfrank@stanford.edu} \\ Department of Psychology \\ Stanford University}

\begin{document}

\maketitle

\begin{abstract}
Children's trouble with scalar implicatures -- inferences from a weaker
lexicalized description that a stronger alternative is true -- is a
puzzle in pragmatic development. Previous research indicates that
children's failures in processing scalar implicatures may be rooted in
an unestablished quantifier scale, with children who struggle with the
quantifier ``some'' being unable to contrast different quantifiers to
make the implicature. However, the source of this failure is unclear.
Here, we explore reaction time as a measure of processing for scalar
implicatures and reasoning about salient alternatives. In our analyses,
we explore overall performance and reaction time patterns across
development, finding that increased reaction times and accuracy for the
quantifiers ``some'' and ``none.'' Motivated by these findings, we use a
Drift Diffusion Model to explore the relationship between accuracy and
reaction time in processing both scalar implicatures, and the
quantifiers ``some'' and ``none'' more broadly. Overall, we find
evidence that while children's performance in scalar implicature tasks
seems to requires additional processing when reasoning about the
quantifier scale.

\textbf{Keywords:}
Pragmatics; development; language.
\end{abstract}

\section{Introduction}\label{introduction}

As listeners attempting to comprehend language, we have available to us
not only verbalized linguistic information, but also the knowledge of
what the speaker \emph{could} have said. In fact, we frequently go
beyond the literal sense of utterances, and use knowledge of these
alternatives to infer a speaker's intended meaning. In the case of
\emph{pragmatic implicatures} (Grice, 1975), a speaker employs a weaker
literal description to imply that a stronger alternative is true. Thus,
an adult listener would strongly infer from the statement ``I enjoyed
\emph{some} of my winter break'' that some (but not \emph{all}) of my
break was pleasant. This \emph{scalar implicature} (SI) relies heavily
on a knowledge of the relevant lexical alternatives in the quantifier
scale \(<\)\emph{none -- some -- all}\(>\), as a listener must be able
to contrast these alternatives to compute the implicature. While scalar
implicatures are easily comprehended by adults, they pose a pragmatic
challenge to children until fairly late in development (Horowitz \&
Frank, 2015; Katsos \& Bishop, 2011; Papafragou \& Musolino, 2003). What
is the source of children's difficulties with scalar implicatures?

One possibile cause of children's scalar implicature failures is their
knowledge of the relevant lexical alternatives, as proposed by the
\emph{Alternatives Hypothesis} (D. Barner \& Bachrach, 2010; David
Barner, Brooks, \& Bale, 2011). For example, this hypothesis predicts
that if children do not have access to ``some,'' they are unable to
directly compare it to ``all'' in making the scalar implicature
\emph{some, but not all.} Across research on the development of scalar
implicature processing, however, children exhibit varying performance,
depending on the paradigm, syntactic construction of the implicature
prompts, access to visual and lexical alternatives, and age (Guasti et
al., 2005; Horowitz \& Frank, 2015; Noveck, 2001; Papafragou \&
Musolino, 2003; Papafragou \& Tantalou, 2004; Stiller, Goodman, \&
Frank, 2014). Due to these varying measures and methods, making
comparisons across these datasets, and empirically testing the
Alternatives Hypothesis was quite difficult.

In an attempt to reconcile these various accounts and test the
Alternatives Hypothesis, Horowitz and Frank (2015) designed a simple
referent selection paradigm that could be used across a broad age range
(3--5 years) to explore lexicalized (scalar) implicatures. In this task,
children saw three book covers, each featuring four familiar objects
(Figure \ref{fig:image}), and the experimenter described a book using
scalar (quantifier) description (e.g., ``On the cover of my book,
\emph{none/some/all} of the pictures are cats.''). Importantly, children
had access to lexical alternatives (over the course of the study), as
well as visual alternatives (within each trial).

With this supportive paradigm, Horowitz and Frank found children
struggled with the quantifiers ``some'' and ``none'' in relation to
``all'', although performance increased across development.
Intriguingly, they found that performance between these two quantifiers
was strongly bimodal and correlated: Children who failed on trials with
the quantifier ``some'' similarly struggled with ``none,'' and vice
versa. While Horowitz \& Frank's (2015) findings did provide some
support for the Alternatives Hypothesis (with children's scalar
implicature performance supported by access to lexical alternatives),
they observed that children were not able to use these alternatives to
improve over the course of the study. Therefore, they hypothesized that
there was another possible cause for children's poor scalar implicature
computation.

These early failures in computing scalar implicatures, and the related
performance processing the quantifiers ``some'' and ``none'' observed in
(Horowitz \& Frank, 2015) presented two hypotheses for sources of
developmental difficulty in this task, namely, lack of quantifier
knowledge and developing inhibitory control (Horowitz, Schneider, \&
Frank, in prep.). We reasoned that the Alternatives Hypothesis
necessitates familiarity with and ability to contrast alternatives on
the quantifier scale \(<\)\emph{none -- some -- all}\(>\); if children's
quantifier knowledge is absent or unestablished, it might lead to
failures in making an implicature. Another possible cause of children's
struggles with scalar implicatures might be that children have complete
quantifier knowledge, but are unable to inhibit an impulse to choose a
more salient alternative (e.g., choosing the book with the most cats
upon hearing ``On the cover of my book, \emph{some} of the pictures are
cats.'')

We explored these two hypotheses in an individual differences task,
running our scalar implicature (Horowitz \& Frank, 2015) task in
conjunction with quantifier-knowledge (D. Barner, Chow, \& Yang, 2009)
and inhibitory control (Zelazo, 2006) tasks. Overall, we found that
children's scalar implicature performance was strongly correlated with
quantifier knowledge, even when controlling for age. While inhibitory
control was strongly correlated with age, it was not related to
children's performance on the SI task (Horowitz et al., in prep.).

Although we found a significant realtionship between quantifier
knowledge and performance on our scalar implicature task, it is likely
that children's difficulties in this task has more than one source.
Importantly, in the course of conducting this individual differences
study, we observed that children who succeeded both in making a scalar
implicature and comprehending ``none'' seemed to have increased response
latencies. This increase in reaction time may be an important indicator
of how children are using and evaluting the salient alternatives.

\begin{CodeChunk}
\begin{figure}[b]

{\centering \includegraphics{figs/image-1} 

}

\caption[Example trial stimuli used in Horowitz and Frank (2015)]{Example trial stimuli used in Horowitz and Frank (2015).}\label{fig:image}
\end{figure}
\end{CodeChunk}

It is possible that children who have a fully-established quantifier
scale must take additional time in using this information to process a
scalar implicature. This hypothesis has some support in previous
literature; Huang \& Snedeker (2009) explored online measures scalar
implicature processing through eye-tracking, and found a delay in
children's location of the referent of a scalar implicature. The exact
relationship between reaction time and accuracy using a supportive
behavioral paradigm, however, is largely unknown. Here, we explore
children's behavioral response latencies in an iPad adaptation of
Horowitz and Frank's (2015) scalar implicature task. In this study, we
explore our hypothesis that computing a scalar implicature might incur
additional processing time for children as they contrast the relevant
lexical alternatives to make a correct decision.

In our analyses, we explore overall accuracy and patterns of
performance, as in (Horowitz \& Frank, 2015; Horowitz et al., in prep.),
and find that children not only struggle in making a scalar implicature,
but also grapple with the quantifier ``none'' until fairly late in
development. In examining reaction time patterns across all quantifier
types, we find evidence of a speed-accuracy tradeoff associated with
these two quantifiers, even later in development. Finally, we use a
Drift Diffusion Model to our data to explore the source of this
increased reaction time. Overall, our findings indicate that while
quantifier knowledge is a key factor in successfully computing scalar
implicatures, using this information to successfully compute a scalar
implicature is particularly difficult, and seems to requires additional
processing.

\section{Method}\label{method}

In this study, we adapted a scalar mplicature paradigm developed by
Horowitz and Frank (2015) for the
iPad.\footnote{The full experiment can be viewed online at \texttt{https://rosemschneider.github.io/tablet\_exp/si\_tablet.html}.}
In addition to capturing detailed reaction time data, this version
included more trials, and standardized prosody across all trials, in
addition to a completely randomized
design.\footnote{All of our data, processing, experimental stimuli, and analysis code can be viewed in the version control repository for this paper at: \texttt{https://github.com/rosemschneider/SI\_tablet}.}

\subsection{Participants}\label{participants}

\begin{table}
\centering
\begin{tabular}{c c c c c } 
 \hline
 Age group & N & Mean & Median & SD \\
 \hline
 3--3.5 years & 22 & 3.27 & 3.27 & 0.15\\
 \hline
 3.5--4 years & 34 & 3.78 & 3.73 & 0.15 \\ 
 \hline
 4--4.5 years & 24 & 4.29 & 4.29 & 0.14\\
 \hline
 4.5--5 years & 30 & 4.76 & 4.76 & 0.15 \\
 \hline
 5--6.5 years & 24 & 5.55 & 5.56 & 0.36 \\
 \hline
\end{tabular}
\caption{Age demographic information for all participants.}
\label{tab:age}
\end{table}

Table \ref{tab:age} shows the breakdown of age information for all
participants. Included in analyses are 134, 134, 134, 134, 134 children
out of a planned sample of 120 participants. We ran 20 additional
children, who were excluded from analysis for low English language
exposure or \$\textless{}\$50\% of trials completed. Included in our
sample were 75 females and 59 males. Based on (Horowitz \& Frank, 2015;
Horowitz et al., in prep.), the initially planned sample size was 96
children from 3--5 years. After collecting data from 57 participants,
however, we observed significantly lower performance on implicature
trials across all age groups, indicating that the iPad adaptation of the
scalar implicature task was slightly more challenging for all children,
and included an older age group of 24 5--6.5-year-olds.

\subsection{Stimuli}\label{stimuli}

The general format of the task was identical to (Horowitz \& Frank,
2015), with the exception of added items for additional trials. The
study was programmed in HTML, CSS, and JavaScript, and displayed to
children on a full-sized iPad. Each trial displayed three book covers,
each containing a set of four familiar objects (Figure \ref{fig:image}).
Each trial allowed 2.5s for children to visually inspect the three book
covers, before the experiment played the trial prompt (e.g., ``On the
cover of my book, \emph{none} of the pictures are cats.''). Each trial
was randomized, with the exception that similar items were displayed
together (e.g., food, clothing). Each session involved 30 trials, with
10 trials per quantifier-type (``all'', ``some'', and ``none''). In our
randomization, quantifier triad order, items (within category), target
item, and quantifier, were randomized for all participants.

\subsection{Procedure}\label{procedure}

\begin{CodeChunk}
\begin{figure}[t]
\includegraphics{figs/overall_acc-1} \caption[Children's overall accuracy for each quantifier type]{Children's overall accuracy for each quantifier type. Bars show mean performance for each age group. Error bars are 95 percent confidence intervals computed by non-parametric bootstrap.}\label{fig:overall_acc}
\end{figure}
\end{CodeChunk}

Sessions took place individually in a small testing room away from
either the museum or the classrooom. To familiarize children to the
iPad, each session began with the ``dot game,'' which required them to
press dots on the screen as fast as possible.

After the dot game, the experimenter introduced them to ``Hannah,'' a
cartoon character who wanted to play a guessing game with her books. The
experimenter explained that Hannah would show the child three books, and
would give \emph{one} hint about which book she had in mind. The
experimenter emphasized that Hannah would only give one hint, so they
had to listen carefully. Children then saw a practice trial with three
books featuring a refrigerator, a TV, and a couch. After 2.5s, a female
voice said ``On the cover of my book there's a TV.'' Once children
correctly made their selection, a green box appeared around the
selection. Children moved trials along at their own pace by pressing a
green button that appeared after they had made their selection.

Reaction times were measured from the onset of the target word. Each
audio clip used the same three frames (e.g., ``On the cover of my book,
\emph{some} of the pictures\ldots{}'') so that prodosdy was emphasized
equally across all trials. The average length of each audio clip
(including target item phrase, e.g., ``\ldots{}are cats'') was
approximately 6s. In all, there were 270 different target items and
audio clips. Children could only make one selection. If a child was not
paying attention, or if she did not hear Hannah's prompt, the
experimenter repeated it, matching the original prosody.

\section{Results}\label{results}

In analyzing the results, we excluded any trials in which reaction time
exceeded fifteen seconds, which indicated that the child had missed the
prompt, or was not paying attention. After this initial cut, we excluded
responses outside three standard deviations of the log of mean reaction
time. This cleaning process resulted in a data loss of 80 trials
(2.07\%). Our planned analyses for this study included explorations of
accuracy, reaction time, and fitting a drift diffusion model (Ratcliff
\& Rouder, 1998) to our reaction time data.

\subsection{Accuracy}\label{accuracy}

In our first planned analysis, we explored children's overall patterns
of accuracy for each quantifier type. Figure \ref{fig:overall_acc} shows
children's for each trial type, split by age group. For each age group,
we saw significantly lower accuracy for the quantifiers ``some'' and
``none'' in comparison to ``all'' in independent t-tests within each age
group (\(p\) \textless{} .01 for all tests). These results replicate our
previous findings using this paradigm (Horowitz \& Frank, 2015; Horowitz
et al., in prep.), indicating that our iPad version was a successful and
appropriate adaptation of the scalar implicature task.

One difference from the previous results was in implicature trials. We
found that children aged 3--5 years performed significantly lower on
``some'' (implicature) trials in this task in comparison with (Horowitz
et al., in prep.) in independent t-tests (\(p\) \textless{} .009 for all
tests). This difference is most likely due to the fact that our
adaptation relied strictly on verbal communication, rather than other
social and nonverbal cues.

\subsubsection{Statistical modeling}\label{statistical-modeling}

\begin{CodeChunk}
\begin{figure}[t]
\includegraphics{figs/diptest-1} \caption[Frequency histogram of correct responses for each trial type, across all participants]{Frequency histogram of correct responses for each trial type, across all participants.}\label{fig:diptest}
\end{figure}
\end{CodeChunk}

\begin{CodeChunk}
\begin{figure*}[t]

{\centering \includegraphics{figs/dense-1} 

}

\caption[Density plots of reaction times for correct and incorrect responses on each trial type, split by age]{Density plots of reaction times for correct and incorrect responses on each trial type, split by age.}\label{fig:dense}
\end{figure*}
\end{CodeChunk}

In exploring children's signficantly lower performance on ``some'' and
``none'' trials, we ran a logistic mixed effects model predicting
correct response as an interaction of age and trial type, with random
effects of trial type and
participant.\footnote{Mixed effects model fit in R using the lme4 package. The model specifications were as follows: \texttt{correct ~ age * trial type + (trial type | subject id)}.}
We found that performance was significantly lower on ``some'' (\(\beta\)
= -7.61, \(p\) \textless{} .0001) and ``none'' trials (\(\beta\) =
-10.09, \(p\) \textless{} .0001). We also found a signficant interaction
between age and trial type on ``none'' trials(\(\beta\) = 1.64, \(p\)
\textless{} .0001), indicating that children's performance with this
difficult quantifier increased with age.

\subsubsection{Correlation between ``some'' and
``none''}\label{correlation-between-some-and-none}

In previous research, a strong correlation has been found on children's
performance with the quantifiers ``some'' and ``none'' (Horowitz \&
Frank, 2015; Horowitz et al., in prep.). Figure \ref{fig:diptest} shows
distributions of correct responses for all trial types. Once again, we
found correlated performance between these two quantifiers (\(r\) =
0.49, \(p\) \textless{} .0001). In running Hartigan's diptest for
bimodality on these two quantifiers, we found significant bimodal
distributions for ``some'' (\(D\) = 0.08, \(p\) \textless{} .0001) and
``none'' trials (\(D\) = 0.11, \(p\) \textless{} .0001). While the
diptest also signficantly rejected unimodality for ``all'' trials, this
is most likely due to the distribution's long tail. The observed
significantly bimodal performance on ``some'' and ``none'' trials,
however, indicate that while children's accuracy is significantly worse
for these quantifiers, they make responses in meaningful manner over the
course of the study. In fact, although children's performance in general
was quite low on these trials, there were a number of children who got
the majority of these trials correct (``Some'': N = 28; ``None'': N =
36).

\subsubsection{Discussion}\label{discussion}

In our accuracy analyses, we found decreased performance across all age
groups for the quantifiers ``some'' and ``none,'' and lower performance
in general for ``some'' (implicature) trials in comparison to previous
results (Horowitz \& Frank, 2015; Horowitz et al., in prep.).
Additionally, we also found bimodal and correlated performance in these
two trial types. Finally, a statistical model revealed a main effects of
age and trial type, as well as significant interactions between age and
trial type, with children becoming more accurate on ``some'' and
``none'' trials with age.

\subsection{Reaction time analyses}\label{reaction-time-analyses}

Response latencies are a largely-unexplored aspect of children's ability
to compute scalar implicatures. We hypothesized that children's reaction
times on this task may reflect processing of a scalar prompt, and
weighing quantifier alternatives. It is possible that this measure may
provide a clue as to the nature of children's correlated struggles with
these terms. In recording reaction times, we began recording from the
onset of the target nouns. Here, we explore overall trends in reaction
times across this task, and the relationship between response latencies
and accuracy on this task.

\subsubsection{Developmental reaction time
distribution}\label{developmental-reaction-time-distribution}

Figure \ref{fig:dense} shows the density of reaction times for each
quantifier, faceted by age group and trial type. Overall, we found that
reaction time was negatively correlated with age (\emph{r} = -0.25,
\emph{p} \textless{} .0001). In exploring the relationship between
accuracy and reaction time, we found preliminary evidence of a
speed-accuracy tradeoff across all trial types. Figure \ref{fig:density}
also shows evidence for increased response latencies associated with
correct responses for the quantifiers ``some'' and ``none''. Overall, we
found that children were largely consistent in their performance across
the course of the study in reaction time correlations between reaction
times for ``all'' and ``some'' (\emph{r} = 0.76, \emph{p} \textless{}
.0001), ``all'' and ``none'' (\emph{r} = 0.66, \emph{p} \textless{}
.0001), and ``some'' and ``none'' trials (\emph{r} = 0.71, \emph{p}
\textless{} .0001).

\begin{CodeChunk}
\begin{figure*}[t]

{\centering \includegraphics{figs/param_plot-1} 

}

\caption[Parameter estimates for drift diffusion model, split by age and trial type]{Parameter estimates for drift diffusion model, split by age and trial type. Error bars are 95 percent confidence intervals computed by nonparametric bootstrap.}\label{fig:param_plot}
\end{figure*}
\end{CodeChunk}

\subsubsection{Statistical modeling}\label{statistical-modeling-1}

We next turned to the relationship betwen age, reaction time, and
accuracy. Our initial hypothesis was that successfully computing
quantifiers might require additional processing time to contrast salient
alternatives. In exploring this, we ran a planned linear mixed effects
model predicting the log of reaction time as an interaction of age,
trial number, and trial type with a random effect of trial
type.\footnote{Mixed effects model fit in R using the lme4 package. The
  model specifications were as follows:
  \texttt{log(reaction time) ~ scale(age) * log(trial number) + scale(age) * trial type + (trial type | subject id)}.
  We calculated \emph{p} values by treating the \emph{t} statistic as if
  it were a \emph{z} statistic Barr, Levy, Scheepers, \& Tily (2013).}
We found a main effect of trial number, with reaction times decreasing
over the course of the study ((\(\beta\) = -0.1, \(p\) \textless{}
.00001)), but found longer reaction times on ``none'' (\(\beta\) = 0.23,
\(p\) \textless{} .00001), and ``some'' trials (\(\beta\) = 0.1, \(p\)
\textless{} .00001). Interestingly, we also found an interaction between
age and trial type, such that reaction times on ``none'' trials
increased with age relative to ``all'' trials (\(\beta\) = -0.02, \(p\)
\textless{} .02), and marginally (but not significantly) increased on
``some'' trials (\(\beta\) = 0.14, \(p\) = .38).

This interaction is particularly intriguing because in our previous
accuracy model, we found increased performance on these trial types.
While we find that older children are taking longer to respond to these
trial types, they are more likely to answer correctly, even though
reaction time is negatively correlated with age. This seems to indicate
that there is a speed-accuracy tradeoff for these quantifiers.

\subsection{Drift diffusion models}\label{drift-diffusion-models}

Our preliminary analysis of children's reaction times in this task
indicated greater response latencies associated with success on ``some''
and ``none'' trials. This suggests that children may be taking more time
to actively compare and contrast scalar alternatives as they become more
familiar with the quantifier scale. A drift diffusion model (DDM) can
provide a more detailed view of the relationship between accuracy and
reaction time in behavioral tasks (Milosavljevic, Malmaud, Huth, Koch,
\& Rangel, 2010; Ratcliff \& Rouder, 1998).

\subsubsection{Parameter estimation}\label{parameter-estimation}

In DDM, a behavioral response (a correct or incorrect choice) is the
result of noisy data accumulation through a diffusion process
(operationalized by response time) (Ratcliff \& Rouder, 1998). Responses
have \emph{separation boundaries} that are dependent on the amount of
information needed to initate a response, and \emph{drift rate}
formalizes the rate of data accumulation (Ratcliff \& Rouder, 1998). An
additional parameter of DDM is \emph{nondecision}, which is the amount
of time between encoding the stimuli, and initiating response. Finally,
different responses may have a \emph{bias}, or different starting point
in the diffusion process, dependent on the stimuli (Ratcliff \& Rouder,
1998).

In fitting a DDM to our data, we estimated parameters for each subject
across all three trial types (``all'', ``some'', and ``none'') using the
RWiener package. We then aggregated across subjects to obtain means and
confidence intervals for each age
group.\footnote{Here we should address reaction time exclusion, when we've made a decision about it.}
Figure \ref{fig:param_plot} shows the parameter estimates for each age
group, split by trial type.

Using the DDM, we can explore in more depth the speed-accuracy tradeoff
we observed in our statistical models. In our separation boundary
estimates, we saw that over development the boundary for a correct
response decreases, for all trial types, indicating that in general less
data needs to be accumulated before making a decision in this task. This
is also reflected in the bias estimates, which are very high for all age
groups on ``all'' trials, but increase over development for ``some'' and
``none'' trials. Interestingly, we found that non-decision decreases
markedly for ``all'' trials over development, but only slightly for
``some'' and ``none'' trials, while drift rates for all quantifiers
increase over development.

In this model, we find that younger children spend less time processing
a quantifier, and are also faster in making an incorrect decision with
the difficult quantifiers ``some'' and ``none''. Young children's high
drift rate and comparable nondecision estimates on ``all'' trials can be
attributed to their tendency to select the alternative for ``all'' on
the majority of trials. Later in development, however, we find that
older children seem to spend more time processing the quantifiers
``some'' and ``none'', and consequently are more likely to make a
correct response. Intriguigingly, we observe this trend even though
older children are have a greater bias and a lower separation boundary
with these quantifier types.

\section{General Discussion}\label{general-discussion}

Our primary question in this study centered on whether children who
succeed on a scalar implicature task (and also understand the
quantifiers involved) take more time to process. We adapted a previously
validated scalar implicature task (Horowitz \& Frank, 2015; Horowitz et
al., in prep.) for the iPad to explore whether children's success how
response latencies interacted with accuracy in this task.

In our analyses, we replicated previous results (Horowitz \& Frank,
2015; Horowitz et al., in prep.) in our finding that children were
overall less accurate when evaluting the quantifiers ``some'' and
``none'' in comparison to ``all,'' but that their performance increased
over development. We again found evidence of bimodal and correlated
performance on these two quantifier types, suggesting a common source of
difficulty.

In our extension of this paradigm, we collected reaction time data for
these quantifier types to investigate the relationship between reaction
time and accuracy. In our reaction time analyses, we found evidence of a
speed-accuracy tradeoff, as well as an interaction between reaction time
and age, with older children taking a slightly longer time to respond to
these trials, but ultimately being more accurate.

Using a Drift Diffusion Model, we explored reaction time and accuracy
patterns in more depth. Overall, the model predicted that children spend
more time processing these difficult pragmatic implicatures as they get
older, and are more likely to make a correct response as a result. Thus,
it appears that success in making a scalar implicature is not only the
result of being familiar with the quantifier scale, but also taking
additional time to process and integrate in a pragmatic framework in
order to make a response.

Our work contributes to the existing literature in utilizing a novel
method to collect accurate and detailed reaction time data on a scalar
implicature task. Response latencies are an important indicator of the
pragmatic challenges that children face in processing implicatures.
Additionally, our findings replicate previous work, providing evidence
for the appropriateness of this paradigm in targeting scalar
implicatures. Further, our larger sample size, increased number of
trials, and randomized design strengthen our analytical power, and allow
for more detailed inferences from the data.

(Final paragraph on quantifier knowledge, inhibitory control, scalar
implicatures, and how they relate more broadly to language).

\section{Acknowledgements}\label{acknowledgements}

Special thanks to Bing Nursery School, the San Jose Children's Discovery
Museum, Veronica Cristiano, Rachel Walker, and Tamara Mekler for their
help with data collection.

\section{References}\label{references}

\setlength{\parindent}{-0.1in} \setlength{\leftskip}{0.125in} \noindent

Barner, D., \& Bachrach, A. (2010). Inference and exact numerical
representation in early language development. \emph{Cognitive
Psychology}, \emph{60}(1), 40--62.

Barner, D., Brooks, N., \& Bale, A. (2011). Accessing the unsaid: The
role of scalar alternatives in children’s pragmatic inference.
\emph{Cognition}, \emph{118}(1), 84--93.

Barner, D., Chow, K., \& Yang, S. (2009). Finding one's meaning: A test
of the relation between quantifiers and integers in language
development. \emph{Cognitive Psychology}, \emph{58}(2), 195--219.

Barr, D. J., Levy, R., Scheepers, C., \& Tily, H. (2013). Random effects
structure for confirmatory hypothesis testing: Keep it maximal.
\emph{Journal of Memory and Language}, \emph{68}(3), 255--278.

Grice, H. (1975). \emph{Logic and conversation} (pp. 41--58).

Guasti, M., Chierchia, G., Crain, S., Foppolo, F., Gualmini, A., \&
Meroni, L. (2005). Why children and adults sometimes (but not always)
compute implicatures. \emph{Language and Cognitive Processes},
\emph{20}(5), 667--696.

Horowitz, A., \& Frank, M. C. (2015). Sources of developmental change in
pragmatic inferences about scalar terms. In \emph{Proceedings of the
37th annual conference of the cognitive science society.}

Horowitz, A., Schneider, R. M., \& Frank, M. C. (in prep.). The trouble
with quantifiers: Children's difficulties with ``some'' and ``none''.

Huang, Y. T., \& Snedeker, J. (2009). Online interpretation of scalar
quantifiers: Insight into the semantics-pragmatics interface.
\emph{Cognitive Psychology}, \emph{58}(3), 376--415.

Katsos, N., \& Bishop, D. (2011). Pragmatic tolerance: Implications for
the acquisition of informativeness and implicature. \emph{Cognition},
\emph{120}(1), 67--81.

Milosavljevic, M., Malmaud, J., Huth, A., Koch, A., \& Rangel, A.
(2010). Drift diffusion model can account for accuracy and reaction time
of value-based choices under high and low time pressure. \emph{Judgment
and Decision Making}, \emph{5}(6), 437--449.

Noveck, I. (2001). When children are more logical than adults:
Experimental investigations of scalar implicature. \emph{Cognition},
\emph{78}(2), 165--188.

Papafragou, A., \& Musolino, J. (2003). Scalar implicatures: Experiments
at the semantics-pragmatics interface. \emph{Cognition}, \emph{86}(3),
253--282.

Papafragou, A., \& Tantalou, N. (2004). Children's computation of
implicatures. \emph{Language Acquisition}, \emph{12}, 71--82.

Ratcliff, R., \& Rouder, J. (1998). Modeling response times for
two-choice decisions. \emph{Psychologial Science}, \emph{9}(5),
347--356.

Stiller, A., Goodman, N., \& Frank, M. (2014). Ad-hoc implicature in
preschool children. \emph{Language Learning and Development},
\emph{11}(2), 176--190.

Zelazo, P. D. (2006). The dimensional change card sort (dCCS): A method
of assessing executive function in children. \emph{Nature Protocols},
\emph{1}, 297--301.

\end{document}
